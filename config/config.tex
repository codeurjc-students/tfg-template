%%%%%%%%%%%%%%%%%%%%%%% Paquetes %%%%%%%%%%%%%%%%%%%%%%%

\usepackage[a4paper,bindingoffset=3mm,bottom=35mm]{geometry}


% Usad \usepackage[dvips]{graphicx} o \usepackage[pdftex]{graphicx} (no ambos)
%\usepackage[dvips]{graphicx} %%% para LaTeX. Las figuras deben estar en formato eps

\usepackage[colorlinks=true,pdftex]{hyperref}   %%% Opcional. Para incluir marcadores y enlaces en el pdf
\usepackage[pdftex]{graphicx}  %%% para pdflatex. Las figuras pueden estar en pdf, jpg, svg y otros formatos


\usepackage[spanish]{babel}

%\usepackage[latin1]{inputenc} % Usad en WinEdt/MikTex
\usepackage[utf8]{inputenc} % Usad en overleaf

%\usepackage[T1]{fontenc}


% Algunos paquetes útiles

\usepackage{amsmath,amssymb}
\usepackage{hyperref}
\usepackage{xcolor}
\usepackage{afterpage}
\usepackage{paralist}
\usepackage{array}
\usepackage{enumerate}
\usepackage{paralist}
\usepackage{enumitem}
\usepackage{float}
\usepackage{setspace}
\usepackage{listings}
\lstset{language=Java,
  showspaces=false,
  showtabs=false,
  breaklines=true,
  showstringspaces=false,
  breakatwhitespace=true,
  commentstyle=\color{pgreen},
  keywordstyle=\color{pblue},
  stringstyle=\color{pred},
  numbers=left,
  stepnumber=1,
  basicstyle=\ttfamily\scriptsize,
  moredelim=**[is][\color{pgrey}]{~}{~}
}
\usepackage{algorithm}
\usepackage{algorithmic}
\usepackage{fancyhdr}
\usepackage{rotating}
\usepackage{multirow}

\definecolor{pblue}{rgb}{0.13,0.13,1}
\definecolor{pgreen}{rgb}{0,0.5,0}
\definecolor{pred}{rgb}{0.9,0,0}
\definecolor{pgrey}{rgb}{0.46,0.45,0.48}

% Otros paquetes

\usepackage{quotchap}
\usepackage{lipsum}

%%%%%%%%%%%%%%%%%%%%%%%%%%%%%%%%%%%%%%%%%%%%%%%%%%%%%%%%






%%%%%%%%%%%%%%%%%%%%%%% Definiciones básicas %%%%%%%%%%%%%%%%%%%%%%%

\newcommand{\escuela}{Escuela Técnica Superior\\de Ingeniería Informática}
\newcommand{\escuelalargo}{Escuela Técnica Superior de Ingeniería Informática}
\newcommand{\universidad}{Universidad Rey Juan Carlos}
\newcommand{\fecha}{Fecha}
\newcommand{\logoUniversidad}{config/logos/logoURJC.pdf} % logoURJC.eps

%%%%%%%%%%%%%%%%%%%%%%%%%%%%%%%%%%%%%%%%%%%%%%%%%%%%%%%%%%%%%%%%%%%%






%%%%%%%%%%%%%%%%%%%%%%%%% Otras definiciones %%%%%%%%%%%%%%%%%%%%%%%%%%

% Definiciones de colores (para hidelinks)
\definecolor{BlueLink}{rgb}{0.165,0.322,0.745}
\definecolor{PinkLink}{rgb}{0.8,0.22,0.5}
\definecolor{gray}{rgb}{0.6,0.6,0.6}


% Enlaces
\hypersetup{hidelinks,pageanchor=true,colorlinks,citecolor=PinkLink,urlcolor=black,linkcolor=BlueLink}


\newcommand\blankpage{%
    \newpage
    \null
    \thispagestyle{empty}%
    %\addtocounter{page}{-1}%
    \newpage}


% Texto referencias
\addto{\captionsspanish}{\renewcommand{\bibname}{Bibliografía}}

% Texto Índice de tablas
\addto\captionsspanish{
\def\tablename{Tabla}
\def\listtablename{\'{I}ndice de tablas}
}


\floatname{algorithm}{Algoritmo}

\newfloat{algorithm}{t}{lop}

%% Etiquetas de comentarios (tutor/alumno)
\newif\ifdraft
\drafttrue
\usepackage{subcaption}
\newcommand{\nb}[2]{
	{
		{\color{black}{
				\small\fbox{\bfseries\sffamily\scriptsize#1}
				{\sffamily\small$\triangleright~${\it\sffamily\small #2}$~\triangleleft$}
	}}}
}

\ifdraft
\newcommand\tutor[1]{\nb{Tutor}{\color{red}#1}}
\newcommand\alumno[1]{\nb{Alumno}{\color{blue}#1}}
\newcommand\cotutor[1]{\nb{Co-tutor}{\color{green}#1}}
\newcommand{\fixme}[1]{{\textcolor{red}{[FIXME] #1}}\xspace}
\newcommand{\cn}{{\color{violet}[citation required]}}

\else
%\usepackage[disable]{todonotes}
\newcommand\tutor[1]{}
\newcommand\alumno[1]{}
\newcommand\cotutor[1]{}
\newcommand{\fixme}[1]{}
\newcommand{\cn}{}

\fi






%\newenvironment{pseudocodigo}[1][htb]
%  {\renewcommand{\algorithmcfname}{Pseudocódig}% Update algorithm name
%   \begin{algorithm}[#1]%
%  }{\end{algorithm}}
  
%%%%%%%%%%%%%%%%%%%%%%%%%%%%%%%%%%%%%%%%%%%%%%%%%%%%%%%%%%%%%%%%%%%%





%%%%%%%%%%%%%%%%%%%%%%% Estilo de código (en Python) %%%%%%%%%%%%%%%%%%%%%%%

\definecolor{bg}{rgb}{0.95,0.95,0.95}
\definecolor{mydeepteal}{rgb}{0.16,0.22,0.23}
\definecolor{myteal}{rgb}{0.31,0.44,0.46}
\definecolor{mymediumteal}{rgb}{0.41,0.58,0.60}

\DeclareFixedFont{\ttb}{T1}{txtt}{bx}{n}{12} % for bold
\DeclareFixedFont{\ttm}{T1}{txtt}{m}{n}{12}  % for normal


%\newcommand*{\FormatDigit}[1]{\textcolor{mydeepteal}{#1}}
\newcommand*{\FormatDigit}[1]{\textcolor{black}{#1}}

% Python style for highlighting
\newcommand\mypythonstyle{\lstset{
language=Python,
basicstyle=\ttfamily\small,
%basicstyle=\linespread{1.0}\footnotesize\ttm,
otherkeywords={self},             % Add keywords here
keywordstyle=\bfseries\ttfamily\color{myteal},
%keywordstyle=\ttb\color{myteal},
commentstyle=\itshape\color{myteal},
stringstyle=\color{mydeepteal},
emph={MyClass,__init__},          % Custom highlighting
emphstyle=\ttb\color{mydeepteal},    % Custom highlighting style
% Any extra options here
showstringspaces=false,            %
backgroundcolor=\color{bg},
rulecolor = \color{bg},
%identifierstyle=\color{deepgreen},
breaklines=true,
numbers=left,
numbersep=5pt,
numberstyle=\tiny,
tabsize=4,
xleftmargin=1em,
frame = single,
framesep = 3pt,
framextopmargin=0pt,
framexbottommargin=0pt,
framexleftmargin=0pt,
framexrightmargin=0pt,
fontadjust=true,
basewidth=0.55em, % compactness of code
upquote=true,
}}

% Python environment
\lstnewenvironment{mypython}[1][]
{
\mypythonstyle
\lstset{#1}
}
{}

\newcommand\mypythonstylenormalinline{\lstset{
language=Python,
basicstyle=\ttfamily\normalsize,
%basicstyle=\linespread{1.0}\footnotesize\ttm,
otherkeywords={self},            % Add keywords here
keywordstyle=\bfseries\ttfamily\color{myteal},
%keywordstyle=\ttb\color{myteal},
commentstyle=\itshape\color{mymediumteal},
stringstyle=\color{mydeepteal},
emph={MyClass,__init__},          % Custom highlighting
emphstyle=\ttb\color{mydeepteal},    % Custom highlighting style
% Any extra options here
showstringspaces=false,            %
backgroundcolor=\color{bg},
rulecolor = \color{bg},
%identifierstyle=\color{deepgreen},
breaklines=false,
numbers=left,
numbersep=5pt,
numberstyle=\tiny,
tabsize=4,
xleftmargin=0em,
frame = single,
framesep = 3pt,
framextopmargin=0pt,
framexbottommargin=0pt,
framexleftmargin=0pt,
framexrightmargin=0pt,
fontadjust=true,
%basewidth=0.55em, % compactness of code
upquote=true,
}}

\newcommand\mypythoninline[1]{{\mypythonstylenormalinline\lstinline!#1!}}

%%%%%%%%%%%%%%%%%%%%%%%%%%%%%%%%%%%%%%%%%%%%%%%%%%%%%%%%%%%%%%%%%%%%%%%%%%%%%%




%%%%%%%%%%%%%%%%%%%%%%%%%%%% Comandos definidos por el autor 

\newcommand{\transpuesta}{\mbox{\tiny $\mathsf{T}$}}