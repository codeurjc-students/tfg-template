% Clase del documento
\documentclass[12pt,twoside,titlepage]{report}





%%%%%%%%%%%%%%%%%%%%%%% Paquetes %%%%%%%%%%%%%%%%%%%%%%%

\usepackage[a4paper,bindingoffset=3mm,bottom=35mm]{geometry}


% Usad \usepackage[dvips]{graphicx} o \usepackage[pdftex]{graphicx} (no ambos)
%\usepackage[dvips]{graphicx} %%% para LaTeX. Las figuras deben estar en formato eps

\usepackage[colorlinks=true,pdftex]{hyperref}   %%% Opcional. Para incluir marcadores y enlaces en el pdf
\usepackage[pdftex]{graphicx}  %%% para pdflatex. Las figuras pueden estar en pdf, jpg, svg y otros formatos


\usepackage[spanish]{babel}

%\usepackage[latin1]{inputenc} % Usad en WinEdt/MikTex
\usepackage[utf8]{inputenc} % Usad en overleaf

%\usepackage[T1]{fontenc}


% Algunos paquetes útiles

\usepackage{amsmath,amssymb}
\usepackage{hyperref}
\usepackage{xcolor}
\usepackage{afterpage}
\usepackage{paralist}
\usepackage{array}
\usepackage{enumerate}
\usepackage{paralist}
\usepackage{enumitem}
\usepackage{float}
\usepackage{setspace}
\usepackage{listings}
\usepackage{algorithm}
\usepackage{algorithmic}
\usepackage{fancyhdr}
\usepackage{rotating}
\usepackage{multirow}


% Otros paquetes

\usepackage{quotchap}
\usepackage{lipsum}

%%%%%%%%%%%%%%%%%%%%%%%%%%%%%%%%%%%%%%%%%%%%%%%%%%%%%%%%






%%%%%%%%%%%%%%%%%%%%%%% Definiciones básicas %%%%%%%%%%%%%%%%%%%%%%%

\newcommand{\nombreautor}{Nombre Apellido1 Apellido2}
\newcommand{\nombretutor}{NombreTutor Apellido1 Apellido2}
\newcommand{\titulotrabajo}{Título del Trabajo de Fin de Grado}
\newcommand{\escuela}{Escuela Técnica Superior\\de Ingeniería Informática}
\newcommand{\escuelalargo}{Escuela Técnica Superior de Ingeniería Informática}
\newcommand{\universidad}{Universidad Rey Juan Carlos}
\newcommand{\fecha}{Fecha}
\newcommand{\grado}{Grado en XXXXXXXX}
\newcommand{\curso}{Curso 20XX-20XX}
\newcommand{\logoUniversidad}{logoURJC.pdf} % logoURJC.eps

%%%%%%%%%%%%%%%%%%%%%%%%%%%%%%%%%%%%%%%%%%%%%%%%%%%%%%%%%%%%%%%%%%%%






%%%%%%%%%%%%%%%%%%%%%%%%% Otras definiciones %%%%%%%%%%%%%%%%%%%%%%%%%%

% Definiciones de colores (para hidelinks)
\definecolor{BlueLink}{rgb}{0.165,0.322,0.745}
\definecolor{PinkLink}{rgb}{0.8,0.22,0.5}
\definecolor{gray}{rgb}{0.6,0.6,0.6}


% Enlaces
\hypersetup{hidelinks,pageanchor=true,colorlinks,citecolor=PinkLink,urlcolor=black,linkcolor=BlueLink}


\newcommand\blankpage{%
    \newpage
    \null
    \thispagestyle{empty}%
    %\addtocounter{page}{-1}%
    \newpage}


% Texto referencias
\addto{\captionsspanish}{\renewcommand{\bibname}{Bibliografía}}

% Texto Índice de tablas
\addto\captionsspanish{
\def\tablename{Tabla}
\def\listtablename{\'{I}ndice de tablas}
}


\floatname{algorithm}{Algoritmo}

\newfloat{algorithm}{t}{lop}

%% Etiquetas de comentarios (tutor/alumno)
\newif\ifdraft
\drafttrue
\usepackage{subcaption}
\newcommand{\nb}[2]{
	{
		{\color{black}{
				\small\fbox{\bfseries\sffamily\scriptsize#1}
				{\sffamily\small$\triangleright~${\it\sffamily\small #2}$~\triangleleft$}
	}}}
}

\usepackage{todonotes}
\ifdraft

% Commands for comments
\newcommand{\tutor}[1]{\todo[color=red!40, inline]{\footnotesize \textbf{Tutor:} #1}}
\newcommand{\alumno}[1]{\todo[color=blue!40, inline]{\footnotesize \textbf{Alumno:} #1}}
\newcommand{\cotutor}[1]{\todo[color=green!40, inline]{\footnotesize \textbf{Co-tutor:} #1}}

\else
\usepackage[disable]{todonotes}
\fi






%\newenvironment{pseudocodigo}[1][htb]
%  {\renewcommand{\algorithmcfname}{Pseudocódig}% Update algorithm name
%   \begin{algorithm}[#1]%
%  }{\end{algorithm}}
  
%%%%%%%%%%%%%%%%%%%%%%%%%%%%%%%%%%%%%%%%%%%%%%%%%%%%%%%%%%%%%%%%%%%%





%%%%%%%%%%%%%%%%%%%%%%% Estilo de código (en Python) %%%%%%%%%%%%%%%%%%%%%%%

\definecolor{bg}{rgb}{0.95,0.95,0.95}
\definecolor{mydeepteal}{rgb}{0.16,0.22,0.23}
\definecolor{myteal}{rgb}{0.31,0.44,0.46}
\definecolor{mymediumteal}{rgb}{0.41,0.58,0.60}

\DeclareFixedFont{\ttb}{T1}{txtt}{bx}{n}{12} % for bold
\DeclareFixedFont{\ttm}{T1}{txtt}{m}{n}{12}  % for normal


%\newcommand*{\FormatDigit}[1]{\textcolor{mydeepteal}{#1}}
\newcommand*{\FormatDigit}[1]{\textcolor{black}{#1}}

% Python style for highlighting
\newcommand\mypythonstyle{\lstset{
language=Python,
basicstyle=\ttfamily\small,
%basicstyle=\linespread{1.0}\footnotesize\ttm,
otherkeywords={self},             % Add keywords here
keywordstyle=\bfseries\ttfamily\color{myteal},
%keywordstyle=\ttb\color{myteal},
commentstyle=\itshape\color{myteal},
stringstyle=\color{mydeepteal},
emph={MyClass,__init__},          % Custom highlighting
emphstyle=\ttb\color{mydeepteal},    % Custom highlighting style
% Any extra options here
showstringspaces=false,            %
backgroundcolor=\color{bg},
rulecolor = \color{bg},
%identifierstyle=\color{deepgreen},
breaklines=true,
numbers=left,
numbersep=5pt,
numberstyle=\tiny,
tabsize=4,
xleftmargin=1em,
frame = single,
framesep = 3pt,
framextopmargin=0pt,
framexbottommargin=0pt,
framexleftmargin=0pt,
framexrightmargin=0pt,
fontadjust=true,
basewidth=0.55em, % compactness of code
upquote=true,
}}

% Python environment
\lstnewenvironment{mypython}[1][]
{
\mypythonstyle
\lstset{#1}
}
{}

\newcommand\mypythonstylenormalinline{\lstset{
language=Python,
basicstyle=\ttfamily\normalsize,
%basicstyle=\linespread{1.0}\footnotesize\ttm,
otherkeywords={self},            % Add keywords here
keywordstyle=\bfseries\ttfamily\color{myteal},
%keywordstyle=\ttb\color{myteal},
commentstyle=\itshape\color{mymediumteal},
stringstyle=\color{mydeepteal},
emph={MyClass,__init__},          % Custom highlighting
emphstyle=\ttb\color{mydeepteal},    % Custom highlighting style
% Any extra options here
showstringspaces=false,            %
backgroundcolor=\color{bg},
rulecolor = \color{bg},
%identifierstyle=\color{deepgreen},
breaklines=false,
numbers=left,
numbersep=5pt,
numberstyle=\tiny,
tabsize=4,
xleftmargin=0em,
frame = single,
framesep = 3pt,
framextopmargin=0pt,
framexbottommargin=0pt,
framexleftmargin=0pt,
framexrightmargin=0pt,
fontadjust=true,
%basewidth=0.55em, % compactness of code
upquote=true,
}}

\newcommand\mypythoninline[1]{{\mypythonstylenormalinline\lstinline!#1!}}

%%%%%%%%%%%%%%%%%%%%%%%%%%%%%%%%%%%%%%%%%%%%%%%%%%%%%%%%%%%%%%%%%%%%%%%%%%%%%%




%%%%%%%%%%%%%%%%%%%%%%%%%%%% Comandos definidos por el autor 

\newcommand{\transpuesta}{\mbox{\tiny $\mathsf{T}$}}








%%%%%%%%%%%%%%%%%%%%%%%%%%%%%%%%%%%%%%%%%%%%%%%%%%%%%%%%%%%%%%%%%%%%%%%
%                           Inicio del documento                       
%%%%%%%%%%%%%%%%%%%%%%%%%%%%%%%%%%%%%%%%%%%%%%%%%%%%%%%%%%%%%%%%%%%%%%%


\begin{document}

\pagestyle{plain}




%%%%%%%%%%%%%%%%%%%%%%%%%%%%%%%%%%%% Portada %%%%%%%%%%%%%%%%%%%%%%%%%%%%%%%%%%

%\pagenumbering{gobble}
%\pagenumbering{arabic}

% Universidad, Facultad
\begin{titlepage}
\selectlanguage{spanish}


% logo
\begin{center}
    \includegraphics[scale=0.7]{\logoUniversidad}
\end{center}

\bigskip

\begin{center}
\begin{LARGE}
\escuela \\
\end{LARGE}
\end{center}

\bigskip
\bigskip

% Grado
\begin{center}
\begin{large}
\textbf{\grado}\\
\end{large}
\end{center}

% Curso
\begin{center}
\begin{large}
\textbf{\curso}\\
\end{large}
\end{center}

\bigskip

\textbf{\begin{center}
\begin{large}
\textbf{Trabajo Fin de Grado}
\end{large}
\end{center}}

\bigskip
\bigskip
\bigskip

% Nombre del TFG
\begin{center}
\textbf{\begin{large}
\MakeUppercase{\titulotrabajo}\\
\end{large}}
\end{center}

% Nombre del autor
\vspace{\fill}
\begin{center}
\textbf{Autor: \nombreautor}\\ \smallskip
% Tutor
\textbf{Tutor: \nombretutor}\\
% Añadir segundo tutor si hubiera


\bigskip

% Fecha
%\textbf{\fecha}\\
\end{center}
\end{titlepage}


%%%%%%%%%%%%%%%%%%%%%%%% Opcional %%%%%%%%%%%%%%%%%%%%%%
%\blankpage

%\thispagestyle{empty}
%\begin{center}

% Nombre del trabajo
%\textbf{\begin{large}
%\MakeUppercase{\titulotrabajo}\\*
%\end{large}}
%\vspace*{0.2cm}
%\vspace{5cm}

% Nombre del autor y del tutor
%\large Autor: \nombreautor \\* \medskip
%\large Tutor: \nombretutor \\*

%\vfill

% Escuela, universidad y fecha
%\escuelalargo \\ \smallskip
%\universidad \\
%\vspace{1cm}
%\fecha \\

%\clearpage

%\end{center}
%%%%%%%%%%%%%%%%%%%%%%%%%%%%%%%%%%%%%%%%%%%%%%%%%%%%%%%%

\hypersetup{pageanchor=true}

\normalsize
\afterpage{\blankpage} % Se deben añadir página en blanco para que lo capítulos de la memoria o estas secciones introductorias empiecen en páginas impares

%%%%%%%%%%%%%%%%%%%%%%%%%%%%%%%%%%%%%%%%%%%%%%%%%%%%%%%%%%%%%%%%%%%%%%%%%%%%%%%





% Estilo de párrafo de los capítulos
\setlength{\parskip}{0.75em}
\renewcommand{\baselinestretch}{1.25}
% Interlineado simple
\spacing{1}

\pagenumbering{Roman}
\setcounter{page}{2}


%%%%%%%%%%%%%%%%%%%%%%%%% Agradecimientos o dedicatoria %%%%%%%%%%%%%%%%%%%%%%%%%%%

\chapter*{Agradecimientos}

Breves agradecimientos o dedicatoria.

\afterpage{\blankpage}

%%%%%%%%%%%%%%%%%%%%%%%%%%%%%%%%%%%%%%%%%%%%%%%%%%%%%%%%%%%%%%%%%%%%%%%%%%%%%%%%%%%






%%%%%%%%%%%%%%%%%%%%%%%%%%%%%%%%%%%% Resumen %%%%%%%%%%%%%%%%%%%%%%%%%%%%%%%%%%%%%%

\chapter*{Resumen}

Breve resumen del Trabajo de Fin de Grado (TFG). Recomendable entre 250-300 palabras, conteniendo los principales objetivos y resultados derivados del mismo.

\mbox{} \bigskip

\noindent \textbf{Palabras clave}:
\begin{compactitem}
    \item Python
    \item Ciberseguridad
    \item Aprendizaje automático (pueden ser varias)
    \item $\ldots$
\end{compactitem}

\afterpage{\blankpage}

%%%%%%%%%%%%%%%%%%%%%%%%%%%%%%%%%%%%%%%%%%%%%%%%%%%%%%%%%%%%%%%%%%%%%%%%%%%%%%%%%%%





%%%%%%%%%%%%%%%%%%%%%%%%%%%%%%%%%%%% Índices %%%%%%%%%%%%%%%%%%%%%%%%%%%%%%%%%%%%

% Estilo de párrafo de los Índices
\setlength{\parskip}{1pt}
\renewcommand{\baselinestretch}{1}
\renewcommand{\contentsname}{Índice de contenidos}


% Índice de contenidos
\tableofcontents
\afterpage{\blankpage}

% Índice de tablas (OPCIONAL)
\listoftables
\afterpage{\blankpage}
\addcontentsline{toc}{chapter}{\noindent \listtablename}

% Índice de figuras (OPCIONAL)
\listoffigures
\afterpage{\blankpage}
\addcontentsline{toc}{chapter}{\listfigurename}

% Índice de códigos/algoritmos (OPCIONAL).   El término "Códigos" se puede cambiar por "Métodos", "Funciones", "Algoritmos", etc.
\renewcommand\lstlistlistingname{Códigos}
\renewcommand\lstlistingname{Código}
\renewcommand\lstlistlistingname{Índice de códigos}

\lstlistoflistings
\afterpage{\blankpage}
\addcontentsline{toc}{chapter}{\lstlistlistingname}


% En este documento (de momento) no se ha considerado incluir un índice de algoritmos/pseudocódigos, como el que aparece en \ref{AdditionalLouvain}

%%%%%%%%%%%%%%%%%%%%%%%%%%%%%%%%%%%%%%%%%%%%%%%%%%%%%%%%%%%%%%%%%%%%%%%%%%%%%%%%%%%





%%%%%%%%%%%%%%%%%%%%%%% Cabeceras y pies de página (Opcional) %%%%%%%%%%%%%%%%%%%%%%%

%\setlength{\headheight}{15.2pt}
\pagestyle{fancy}


\renewcommand{\chaptermark}[1]{\markboth{Capítulo \thechapter.\ #1}{}}

\pagestyle{fancy}
\fancyhf{}
\fancyhead[LO]{\leftmark}
\fancyhead[RO]{}
\fancyhead[RE]{\nouppercase\rightmark}
\fancyhead[LE]{}
\fancyfoot[C]{\thepage}

%%%%%%%%%%%%%%%%%%%%%%%%%%%%%%%%%%%%%%%%%%%%%%%%%%%%%%%%%%%%%%%%%%%%%%%%%%%%%%%%%%%%






%%%%%%%%%%%%%%%%%%%%%%%%%%%%%% Capítulos de la memoria %%%%%%%%%%%%%%%%%%%%%%%%%%%%%



% Capítulo 1
\chapter{Introducción}


%%%%%%%%%%%%%%%%%%%%%%%%%%%%%%%%%%%%%%%%%%%%%%%%%%%%%%%%%%%%%%%%%%%%%%%%%%

% Estilo resto de páginas
\pagestyle{fancy}


% Estilo de párrafo de los capítulos
\setlength{\parskip}{0.75em}
\renewcommand{\baselinestretch}{1.25}
% Interlineado simple
\spacing{1}
% Numeración contenido
\pagenumbering{arabic}
\setcounter{page}{1}

%%%%%%%%%%%%%%%%%%%%%%%%%%%%%%%%%%%%%%%%%%%%%%%%%%%%%%%%%%%%%%%%%%%%%%%%%%

Se puede añadir texto antes de empezar la primera sección.


\section{Contexto y alcance}

Contexto. Situar al lector. Objetivo general y alcance del trabajo.


\section{Estructura del documento}

La estructura del TFG no es fija. El tutor indicará una estructura adecuada dependiendo del trabajo concreto.\tutor{Comentario del tutor}

Se puede incluir dentro de cada apartado secciones adicionales. La copia en papel de la memoria del TFG será encuadernada en pasta dura de color azul (p.e. encuadernación tipo chanel). La portada, que puede ser una pegatina transparente, seguirá el modelo que se adjunta, que incluye el escudo y nombre de la URJC, la titulación cursada por el alumno, el curso académico, el título del TFG, el autor y el o los directores/tutores.\alumno{Comentario del alumno}


\subsection{Trabajos de grados en informática}

Una posible estructura de la memoria final asociada con cada TFG podría ser la siguiente (leed la normativa de TFG):
\begin{enumerate}
 \item Introducción
 \item Objetivos (incluyendo descripción del problema, estudio de alternativas y metodología empleada)
 \item Descripción informática (puede incluir especificación, diseño, implementación y pruebas).
 \item Experimentos / validación
 \item Conclusiones (incluyendo los logros principales alcanzados y posibles trabajos futuros)
 \item Bibliografía
 \item Apéndices
\end{enumerate}


\subsection{Trabajos del grado en matemáticas}

Una posible estructura de la memoria final asociada con cada TFG podría ser la siguiente:
\begin{enumerate}
 \item Introducción
 \item Objetivos (incluyendo descripción del problema, estudio de alternativas y metodología empleada)
 \item Material y métodos / Metodología / Cuerpo del trabajo (describir las metodologías empleadas en el desarrollo del TFG o el desarrollo del mismo en caso de ser un trabajo de recopilación bibliográfica sobre un tema).
 \item Resultados (opcional, dependiendo del tipo de trabajo desarrollado)
 \item Conclusiones (incluyendo los logros principales alcanzados y posibles trabajos futuros)
 \item Bibliografía
 \item Apéndices
\end{enumerate}

% \afterpage{\blankpage} % puede generar problema en índice de contenidos
% \newpage


% Capítulo 2
\chapter{Objetivos}

Una página describiendo los objetivos concretos que se pretenden conseguir con el desarrollo del proyecto. Es como la conclusión del capítulo anterior.


\blankpage

% Capítulo 3
\chapter{Tecnologías, Herramientas y Metodologías}
\label{chap:tecnologias}

Descripción de los lenguajes de programación, entornos de desarrollo, herramientas auxiliares, librerías de terceros, sistemas operativos, navegadores web, etc… utilizados para la realización del proyecto así como la metodología empleada. 
El grado de profundidad a la hora de explicar cada tecnología dependerá de lo relevante que ha sido para el proyecto y lo conocida que es. 
Por ejemplo, si se usa el lenguaje de programación Java, no es necesario entrar en tanto detalle que si se usa un lenguaje mucho menos usado como Scala, por ejemplo. Respecto a la metodología, dada la naturaleza de los proyectos, se suele describir una metodología iterativa e incremental en espiral, en la que se van sucediendo reuniones con el profesor que van definiendo el ámbito del proyecto. Este capítulo puede tener una extensión entre 10 y 15 páginas.

\section{Tecnologías}

\subsection{Java}

\section{Herramientas}

\section{Metodologías}



% Capítulo 4
\chapter{Descripción Informática}
\label{sec:descripcionInformatica}

Este capítulo recoge todos los aspectos informáticos de la aplicación. 
\tutor{Sección "Guía de desarrollo" de la documentación de repositorio salvo el subcapítulo 4.4, que debe crearse nuevo} 

Comenzar indicando dónde se puede encontrar el repositorio con el código de la aplicación (enlace a GitHub)

Párrafo indicando la arquitectura de despliegue de la aplicación web. Habitualmente se especificará que se trata de una aplicación web con arquitectura SPA (y habrá que explicar en qué consiste) y las partes que tiene (cliente, servidor y base de datos). Si es una aplicación distribuida se hará una descripción de alto nivel de este aspecto. 

Resumen de esta descripción en formato tabla/lista: 

\begin{itemize}
    \item Tipo: web MVC, Web SPA, API REST, microservicios... 

    \item Tecnologías: lenguajes, librerías, servicios adicionales. 

    \item Herramientas: IDEs empleados y herramientas auxiliares. 

    \item Control de calidad: Qué controles de calidad se aplican y qué tecnologías/herramientas se usan. 

    \item Despliegue: Cómo se empaqueta, distribuye y despliega la aplicación (Docker, Kubernetes). Entorno de despliegue (si aplica en este TFG). 

    \item Proceso de desarrollo: iterativa e incremental, git, DevOps (CI/CD). 
\end{itemize}

Párrafo describiendo cada una de las secciones en las que se divide el capítulo (que se indican a continuación) 

\section{Tecnologías}

\tutor{Extensión de la sección: 2/3 páginas. (salvo que haya alguna tecnología de partes optativas que necesite describirse en detalle)}

Tecnologías que usa la aplicación para su ejecución (no las herramientas usadas para su desarrollo).  

\begin{itemize}
    \item Sólo se profundizará si se consideran poco conocidas. Si no, serán de un párrafo. 

    \item Si no es evidente se indicará para qué se usan en el proyecto. 

    \item En todas se indicará la URL oficial.  
\end{itemize}
 

\section{Herramientas}

\tutor{Extensión de la sección: 2/3 páginas}

IDEs empleados y herramientas auxiliares.  

\begin{itemize}
    \item Sólo se profundizará si se consideran poco conocidas. Si no, serán de un párrafo. 

    \item Si no es evidente se indicará para qué se usan en el proyecto. 

    \item En todas se indicará la URL oficial. 
\end{itemize}

\section{Arquitectura}

\tutor{Extensión de la sección: 5/6 páginas}

\subsection{Despliegue}

Arquitectura de despliegue (indicando procesos independientes y protocolos de comunicación) 

\subsection{Modelo del dominio} 

Entidades persistentes de la aplicación, sus atributos y sus relaciones.  

\subsection{API REST} 
Se describirá en alto nivel la API REST como un listado de los recursos y sus operaciones. Para mantener la documentación clara se recomienda dedicar una subsección a cada recurso disponible, indicando:

\begin{itemize}
    \item Formato JSON del recurso (respuestas de consulta o detalles del recurso).
    \item Operaciones disponibles (endpoints) agrupadas por método HTTP.
\end{itemize}

\subsubsection{Recurso Books}
\begin{lstlisting}[basicstyle=\jsonbasicfont,frame=single,backgroundcolor=\color{white},rulecolor=\color{black},framerule=0.5pt,numbers=none]
{
  "id": 1,
  "title": "Clean Architecture",
  "author": "Robert C. Martin",
  "isbn": "9780134494166"
}
\end{lstlisting}

\textbf{Operaciones}
\begin{itemize}
    \item \textbf{GET /api/books/\{id\}} - Obtiene la información de un libro específico por su identificador.
    \item \textbf{GET /api/books} - Obtiene el listado completo de libros.
    \item \textbf{POST /api/books} - Crea un nuevo libro en el sistema.
    \item \textbf{PUT /api/books/\{id\}} - Actualiza la información completa de un libro existente.
    \item \textbf{PATCH /api/books/\{id\}} - Actualiza parcialmente la información de un libro existente.
    \item \textbf{DELETE /api/books/\{id\}} - Elimina un libro del sistema.
\end{itemize}

\tutor{Los alumnos deben documentar todos los recursos de la API siguiendo este esquema. Si existen más recursos (usuarios, préstamos, pedidos, etc.), se repetirá la estructura para cada uno.}

\subsection{Arquitectura del servidor} 
Diagrama de clases del servidor reflejando su separación por capas. Responsabilidades de cada capa (Controladores, servicios, repositorios...).

\subsection{Arquitectura del cliente} 
Diagrama de clases del cliente reflejando su separación por capas. Responsabilidades de cada capa (Componentes, servicios...).

\section{Implementación} 

\tutor{Extensión de la sección: 5/12 páginas}

Aspectos relevantes de la implementación. 

\tutor{Esta sección debe crearse nueva}
 
Por ejemplo, se podría incluir alguno de los siguientes aspectos: 

\begin{itemize}
    \item Algoritmo complejo que se haya tenido que desarrollar. 

    \item Integración entre librerías con la que hayan surgido problemas. 

    \item Resolución de algún bug que haya sido especialmente problemático. 

    \item Estudio que haya sido necesario para elegir una tecnología.  

    \item Focalizar en alguna parte de la lógica de negocio que haya sido especialmente compleja y describirla en más detalle. 
\end{itemize}

En esta sección se pueden incluir fragmentos de código fuente (ver Anexo~\ref{anexo:latex} para ejemplos de cómo incluirlos).

\section{Control de calidad} 

\tutor{Extensión de la sección: 3/4 páginas}

Descripción de los controles de calidad que se han realizado.  

Descripción de las pruebas automáticas de cliente y servidor:  

\begin{itemize}
    \item Tipos de pruebas 

    \item Descripción de qué funcionalidades se prueban. Ya que están numeradas en el anexo, conviene que haya algún tipo de trazabilidad entre la prueba y la funcionalidad. 

    \item Estadísticas de las pruebas (número, cobertura, etc.). Mostrar captura de pantalla de su ejecución. 
\end{itemize}

Herramientas de análisis estático de código (si se ha usado): 

\begin{itemize}
    \item Captura de pantalla de los resultados de los análisis de código finales o a lo largo del tiempo. 
    \item Métricas del tamaño del código (número de clases, número de líneas de código, separadas por tecnologías...) 
\end{itemize}

\section{Despliegue}
\tutor{Extensión de la sección: 2/3 páginas}

Cómo se realiza el empaquetado, distribución y despliegue

\begin{itemize}
    \item Empaquetado y distribución: una única imagen Docker para cliente y servidor, docker compose para coordinar, uso de Kubernetes, serverless...). Se indicará la URL para acceder al artefacto de la aplicación (DockerHub). 

    \item Despliegue: Computación en la nube (AWS, Azure...), tecnología Infraestructura como código (Cloud Formation...). 
\end{itemize}

\section{Proceso de desarrollo} 
\tutor{Extensión de la sección: 4/5 páginas}

Descripción de los aspectos técnicos del proceso de desarrollo (de las fases 2 a 5).
 
Proceso iterativo e incremental, que sigue los principios del manifiesto ágil y se apoya en algunas de las buenas prácticas de Programación Extrema (XP) y Kanban. No se puede decir que se ha aplicado Scrum, porque no se aplica. 
 

\subsection{Gestión de tareas} 
GitHub Issues, GitHub Projects y gestión visual (tablero).

\subsection{Git} 
Se describe que se ha usado un repositorio git y la estrategia de ramas utilizada. Métricas de uso de git: Número de commits, número de ramas, etc.

\subsection{Integración y entrega continua} 
Se indican las tareas que realizan los flujos automáticos del sistema de integración continua (workflows de GitHub Actions).

\subsection{Despliegue continuo y/o versionado} 
Descripción del procedimiento de lanzamiento de versiones (releases), fechas en las que se ha publicado cada versión, descripción de alto nivel de las funcionalidades que incluye, etc. Si se ha usado Despliegue continuo, debería definirse aquí y cómo se relaciona con las versiones. Se debería mencionar el impacto en los usuarios cuando se produce el despliegue.

\blankpage

% Capítulo 5

\chapter{Conclusiones y trabajos futuros}

Reflexión sobre el trabajo realizado, qué objetivos se han cumplido y qué aspectos quedan pendientes para una futura ampliación del proyecto. Además, se deben incluir unas conclusiones personales indicando lo que ha supuesto para el alumno la realización del trabajo. Entre 2 y 4 páginas.

\blankpage


%%%%%%%%%%%%%%%%%%%%%%%%%%%%%%% Bibliografía %%%%%%%%%%%%%%%%%%%%%%%%%%%%%%%

\phantomsection
\addcontentsline{toc}{chapter}{Bibliografía}

\footnotesize{
%\bibliographystyle{hispa}
\bibliographystyle{IEEEtran}
\bibliography{bibliografia}
}



% No expandir elementos para llenar toda la página
\raggedbottom
\afterpage{\blankpage}

\newpage




%%%%%%%%%%%%%%%%%%%%%%%%%%%%%%% Apéndices %%%%%%%%%%%%%%%%%%%%%%%%%%%%%%%

\appendix

\phantomsection
\addcontentsline{toc}{chapter}{Apéndices}

\mbox{}
\vfill
\begin{center}
\begin{Huge}
\textbf{Apéndices}
\end{Huge}
\end{center}
\vfill
\mbox{}
\thispagestyle{empty}

\newpage
\mbox{}
\thispagestyle{empty}
\newpage


% Primer apéndice
\chapter{Este es el primer apéndice}
\label{sec:apendice}

\section{Ejemplo de sección}

Sección del apéndice


% Fin del documento
\end{document}
