\tutor{Extensión del capítulo: 4/8 páginas}

Contexto en el que se enmarca el proyecto y la justificación de este.  

\section{Contexto}
Contexto en el que se enmarca el proyecto y la justificación de este. Esta sección explica la utilidad de la aplicación, el contexto en el que sería usada y las funcionalidades principales que tendría y por qué es relevante su implementación, etc. No se especifican aspectos técnicos. \tutor{Debe crearse nueva}

\section{Estado del arte}
Descripción de otras aplicaciones y plataformas web que ofrecen funcionalidad parecida a la que se ha implementado y se destaca en qué aspectos es diferente tu página. \tutor{Debe crearse nueva}

\section{Objetivos}
Objetivos que se plantearon cuando se inició este trabajo. \tutor{Sección “Objetivos” de la documentación de repositorio}


\subsection{Objetivos funcionales} 

Un párrafo resumiendo los objetivos funcionales que se plantearon al comenzar el desarrollo de la aplicación. 

Una lista de 3-10 funcionalidades detallando un poco más el párrafo de objetivos funcionales. 

\begin{itemize}
    \item Objetivo funcional 1
    \item Objetivo funcional 2
    \item Objetivo funcional 3
    \item ...
\end{itemize}

\subsection{Objetivos técnicos}

Un párrafo resumiendo los aspectos tecnológicos que se plantearon al comenzar el desarrollo de la aplicación. 

Una lista de 3-10 aspectos técnicos detallando un poco más el párrafo de objetivos técnicos. 

\begin{itemize}
    \item Objetivo técnico 1
    \item Objetivo técnico 2
    \item Objetivo técnico 3
    \item ...
\end{itemize}