Descripción de los lenguajes de programación, entornos de desarrollo, herramientas auxiliares, librerías de terceros, sistemas operativos, navegadores web, etc… utilizados para la realización del proyecto así como la metodología empleada. 
El grado de profundidad a la hora de explicar cada tecnología dependerá de lo relevante que ha sido para el proyecto y lo conocida que es. 
Por ejemplo, si se usa el lenguaje de programación Java, no es necesario entrar en tanto detalle que si se usa un lenguaje mucho menos usado como Scala, por ejemplo. Respecto a la metodología, dada la naturaleza de los proyectos, se suele describir una metodología iterativa e incremental en espiral, en la que se van sucediendo reuniones con el profesor que van definiendo el ámbito del proyecto. Este capítulo puede tener una extensión entre 10 y 15 páginas.

\section{Tecnologías}

\subsection{Java}

\section{Herramientas}

\section{Metodologías}
