Descripción del proyecto realizado. Después de unos párrafos introductorios el capítulo se divide en subcapítulos. (de 25 a 35 páginas)

\section{Requisitos}

Descripción detallada de las funcionalidades que tendría que implementar la aplicación (pues se asume que los requisitos se escriben antes de empezar el desarrollo). Pueden tener forma de historias de usuario o bien ser una lista de requisitos funcionales y no funcionales.

Ejemplo:

En esta sección abordaremos los requisitos planteados en la aplicación, tanto los iniciales como los que se han ido planteando a lo largo del desarrollo. 
También se nombrarán otro tipo de requisitos a la hora de desarrollar el proyecto que no están relacionados en sí con el propio funcionamiento de la aplicación.

\subsection{Requisitos Funcionales}

\begin{itemize}
    \item RF1. Como usuario puedo ...
\end{itemize}

\subsection{Requisitos No Funcionales}

\begin{itemize}
    \item RNF1. La aplicación web debe ser fácil de usar ...
\end{itemize}

\section{Arquitectura y Análisis}

Descripción de los aspectos de alto nivel de la aplicación. Diagramas de clases de análisis, diagramas de clases de diseño, etc. Se debe incluir la suficiente información para que el lector pueda entender la estructura de alto nivel del software desarrollado. Se pueden incluir diagramas de casos de uso si se considera útil.

\section{Diseño e Implementación} 

Descripción de algún aspecto relevante de la implementación que quiera mencionarse. Por ejemplo se podría incluir alguno de los siguientes aspectos:
\begin{itemize}
    \item Algoritmo complejo que se haya tenido que desarrollar.
    \item Integración entre librerías problemática.
    \item Resolución de algún bug que haya sido especialmente problemático.
    \item Focalizar en alguna parte del desarrollo y describirla en más detalle
\end{itemize}

En esta sección se pueden incluir fragmentos de código fuente. En este apartado se pueden incluir algunas métricas del proyecto (Nº de clases, líneas de código, etc…). También se puede incluir la evolución del repositorio de github (gráfico de commits por día).

\section{Pruebas} 

En esta sección se describen las pruebas automáticas que han sido implementadas para el proyecto. Sobre los tests, conviene indicar la cobertura del código. Si no se han implementado pruebas automáticas, deberían haberse implementarse y describirse aquí o tener una buena justificación de por qué no se han implementado.

\section{Distribución y despliegue} 

En esta sección se hablará de cómo se ha empaquetado la aplicación en Docker, cómo se ha desplegado, etc.
