Este capítulo recoge todos los aspectos informáticos de la aplicación. 
\tutor{Sección "Guía de desarrollo" de la documentación de repositorio salvo el subcapítulo 4.4, que debe crearse nuevo} 

Comenzar indicando dónde se puede encontrar el repositorio con el código de la aplicación (enlace a GitHub)

Párrafo indicando la arquitectura de despliegue de la aplicación web. Habitualmente se especificará que se trata de una aplicación web con arquitectura SPA (y habrá que explicar en qué consiste) y las partes que tiene (cliente, servidor y base de datos). Si es una aplicación distribuida se hará una descripción de alto nivel de este aspecto. 

Resumen de esta descripción en formato tabla/lista: 

\begin{itemize}
    \item Tipo: web MVC, Web SPA, API REST, microservicios... 

    \item Tecnologías: lenguajes, librerías, servicios adicionales. 

    \item Herramientas: IDEs empleados y herramientas auxiliares. 

    \item Control de calidad: Qué controles de calidad se aplican y qué tecnologías/herramientas se usan. 

    \item Despliegue: Cómo se empaqueta, distribuye y despliega la aplicación (Docker, Kubernetes). Entorno de despliegue (si aplica en este TFG). 

    \item Proceso de desarrollo: iterativa e incremental, git, DevOps (CI/CD). 
\end{itemize}

Párrafo describiendo cada una de las secciones en las que se divide el capítulo (que se indican a continuación) 

\section{Tecnologías}

\tutor{Extensión de la sección: 2/3 páginas. (salvo que haya alguna tecnología de partes optativas que necesite describirse en detalle)}

Tecnologías que usa la aplicación para su ejecución (no las herramientas usadas para su desarrollo).  

\begin{itemize}
    \item Sólo se profundizará si se consideran poco conocidas. Si no, serán de un párrafo. 

    \item Si no es evidente se indicará para qué se usan en el proyecto. 

    \item En todas se indicará la URL oficial.  
\end{itemize}
 

\section{Herramientas}

\tutor{Extensión de la sección: 2/3 páginas}

IDEs empleados y herramientas auxiliares.  

\begin{itemize}
    \item Sólo se profundizará si se consideran poco conocidas. Si no, serán de un párrafo. 

    \item Si no es evidente se indicará para qué se usan en el proyecto. 

    \item En todas se indicará la URL oficial. 
\end{itemize}

\section{Arquitectura}

\tutor{Extensión de la sección: 5/6 páginas}

\subsection{Despliegue}

Arquitectura de despliegue (indicando procesos independientes y protocolos de comunicación) 

\subsection{Modelo del dominio} 

Entidades persistentes de la aplicación, sus atributos y sus relaciones.  

\subsection{API REST} 
Se describirá en alto nivel la API REST como un listado de los recursos y sus operaciones. Para mantener la documentación clara se recomienda dedicar una subsección a cada recurso disponible, indicando:

\begin{itemize}
    \item Formato JSON del recurso (respuestas de consulta o detalles del recurso).
    \item Operaciones disponibles (endpoints) agrupadas por método HTTP.
\end{itemize}

\subsubsection{Recurso Books}
\begin{lstlisting}[basicstyle=\jsonbasicfont,frame=single,backgroundcolor=\color{white},rulecolor=\color{black},framerule=0.5pt,numbers=none]
{
  "id": 1,
  "title": "Clean Architecture",
  "author": "Robert C. Martin",
  "isbn": "9780134494166"
}
\end{lstlisting}

\textbf{Operaciones}
\begin{itemize}
    \item \textbf{GET /api/books/\{id\}} - Obtiene la información de un libro específico por su identificador.
    \item \textbf{GET /api/books} - Obtiene el listado completo de libros.
    \item \textbf{POST /api/books} - Crea un nuevo libro en el sistema.
    \item \textbf{PUT /api/books/\{id\}} - Actualiza la información completa de un libro existente.
    \item \textbf{PATCH /api/books/\{id\}} - Actualiza parcialmente la información de un libro existente.
    \item \textbf{DELETE /api/books/\{id\}} - Elimina un libro del sistema.
\end{itemize}

\tutor{Los alumnos deben documentar todos los recursos de la API siguiendo este esquema. Si existen más recursos (usuarios, préstamos, pedidos, etc.), se repetirá la estructura para cada uno.}

\subsection{Arquitectura del servidor} 
Diagrama de clases del servidor reflejando su separación por capas. Responsabilidades de cada capa (Controladores, servicios, repositorios...).

\subsection{Arquitectura del cliente} 
Diagrama de clases del cliente reflejando su separación por capas. Responsabilidades de cada capa (Componentes, servicios...).

\section{Implementación} 

\tutor{Extensión de la sección: 5/12 páginas}

Aspectos relevantes de la implementación. 

\tutor{Esta sección debe crearse nueva}
 
Por ejemplo, se podría incluir alguno de los siguientes aspectos: 

\begin{itemize}
    \item Algoritmo complejo que se haya tenido que desarrollar. 

    \item Integración entre librerías con la que hayan surgido problemas. 

    \item Resolución de algún bug que haya sido especialmente problemático. 

    \item Estudio que haya sido necesario para elegir una tecnología.  

    \item Focalizar en alguna parte de la lógica de negocio que haya sido especialmente compleja y describirla en más detalle. 
\end{itemize}

En esta sección se pueden incluir fragmentos de código fuente (ver Anexo~\ref{anexo:latex} para ejemplos de cómo incluirlos).

\section{Control de calidad} 

\tutor{Extensión de la sección: 3/4 páginas}

Descripción de los controles de calidad que se han realizado.  

Descripción de las pruebas automáticas de cliente y servidor:  

\begin{itemize}
    \item Tipos de pruebas 

    \item Descripción de qué funcionalidades se prueban. Ya que están numeradas en el anexo, conviene que haya algún tipo de trazabilidad entre la prueba y la funcionalidad. 

    \item Estadísticas de las pruebas (número, cobertura, etc.). Mostrar captura de pantalla de su ejecución. 
\end{itemize}

Herramientas de análisis estático de código (si se ha usado): 

\begin{itemize}
    \item Captura de pantalla de los resultados de los análisis de código finales o a lo largo del tiempo. 
    \item Métricas del tamaño del código (número de clases, número de líneas de código, separadas por tecnologías...) 
\end{itemize}

\section{Despliegue}
\tutor{Extensión de la sección: 2/3 páginas}

Cómo se realiza el empaquetado, distribución y despliegue

\begin{itemize}
    \item Empaquetado y distribución: una única imagen Docker para cliente y servidor, docker compose para coordinar, uso de Kubernetes, serverless...). Se indicará la URL para acceder al artefacto de la aplicación (DockerHub). 

    \item Despliegue: Computación en la nube (AWS, Azure...), tecnología Infraestructura como código (Cloud Formation...). 
\end{itemize}

\section{Proceso de desarrollo} 
\tutor{Extensión de la sección: 4/5 páginas}

Descripción de los aspectos técnicos del proceso de desarrollo (de las fases 2 a 5).
 
Proceso iterativo e incremental, que sigue los principios del manifiesto ágil y se apoya en algunas de las buenas prácticas de Programación Extrema (XP) y Kanban. No se puede decir que se ha aplicado Scrum, porque no se aplica. 
 

\subsection{Gestión de tareas} 
GitHub Issues, GitHub Projects y gestión visual (tablero).

\subsection{Git} 
Se describe que se ha usado un repositorio git y la estrategia de ramas utilizada. Métricas de uso de git: Número de commits, número de ramas, etc.

\subsection{Integración y entrega continua} 
Se indican las tareas que realizan los flujos automáticos del sistema de integración continua (workflows de GitHub Actions).

\subsection{Despliegue continuo y/o versionado} 
Descripción del procedimiento de lanzamiento de versiones (releases), fechas en las que se ha publicado cada versión, descripción de alto nivel de las funcionalidades que incluye, etc. Si se ha usado Despliegue continuo, debería definirse aquí y cómo se relaciona con las versiones. Se debería mencionar el impacto en los usuarios cuando se produce el despliegue.